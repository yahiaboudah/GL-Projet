
\section{Présentation}
\raggedright\setlength{\parindent}{1em}
Dans un environnement universitaire dynamique, la gestion efficace des accès devient une nécessité. Ce projet vise à introduire un système de contrôle d'accès physique aux entrées de l'université, en utilisant des tourniquets tripodes et des codes QR. Ce système implique deux types d'employés: un agent et un administrateur. L'université doit faire face à diverses difficultés concernant la sécurité et la gestion des entrées. Parmi celles-ci, la nécessité de prévenir les entrées répétées avec le même code QR, d'organiser les entrées de manière structurée et de limiter l'accès exclusivement aux membres de l'université.
\newline
\newline
Notre objectif dans ce document est de réaliser une solution en utilisant la méthodologie UML. Le processus est simple: un membre de l'université présente son code QR au tourniquet, qui le scanne pour valider l'accès. Si le code est valide, l'entrée est autorisée, sinon elle est refusée. Les membres ont l'option de demander de l'assistance à l'agent pour entrer si ils ont oublié leur code, ou si leur code ne marche pas. Ils ont aussi l'option de mettre à jour leur code s'ils soupçonnent que quelqu'un a volé une version de leur code.
Notre agent attrape les entrants illégaux lorsqu'ils tentent de sortir et les oblige à payer une amende. L'admin peut réaffecter les agents à différentes portes et contrôler les tourniquets.
\newline
\newline
En adoptant ce système, nous visons à renforcer la sécurité, à organiser les entrées et à garantir que seuls les membres de l'université bénéficient d'un accès autorisé.
\newline
\newline
Ce projet a connu de nombreuses itérations, présentées dans la conclusion du projet.