

TODO
\section{Étape Post-UML}
    TODO
\section{Itérations du mini projet}
    \subsection{Itération Nº1}

        \textbf{\textit{Entités Générales}}
        \vspace{8px}
        \linebreak
        Université; Personne; Tourniquet; Porte; CodeQR;
        Enseignat, Etudiant, Staff (sous-entités de Personne);Alerte; TypeAlerte; AgentSupportTechnique, AgentSécurité (sous-entités de AgentTourniquet); Passage (Entrée ou Sortie); SalleSurveillance; Imprimante; ScannerQRCode; OrdinateurSalleSurveillance
        \linebreak

        \textbf{\textit{Scénario Principal}}
        \vspace{8px}

        \begin{itemize}
            \item Une personne entre à l'université par une porte équipée de tourniquets. Il présente sa carte ou son QR CODE via une application universitaire.
            \item Le tourniquet vérifie la validité du code. Si valide, l'entrée est autorisée ; sinon, une alerte est déclenchée.
            \item L'agent technique informe l'agent sécurité du nombre de tourniquet concerné.
            \item L'agent de sécurité amène une personne dans la salle de surveillance
            \item Si l'alerte est: CodeInconnu, donc l'agent de sécurité retire la personne de l'entrée.
            \item Si l'alerte est: CodeAncien, l'agent technique vérifie si c'est la meme personne, si c'est le cas il lui imprime le code le plus récent. Sinon la personne est retirée.
            \item Si l'alerte est de type: CodeRépété. Si c'est pas la meme personne dans le système, la personne est retirée, puis l'agent technique met à jour le code de Personne propriétaire de code, et le notifie que son code a été utilisé par quelqu'un d'autre. Si c'est la meme personne, donce son code est mis à jour, et imprimé pour son entrée.
            \item Si le visiteur oublie sa carte ou n'a pas l'application, il se rend à la salle de surveillance.
            \item L'agent technique vérifie l'identité du visiteur dans la base de données. S'il est membre de l'université, un QR CODE temporaire est imprimé pour lui permettre l'accès moyennant une amende.
            \item En sortant, le visiteur glisse sa carte dans le tourniquet de sortie.
            \item Pour éviter les entrées multiples avec le même code QR, un suivi des entrées/sorties est effectué.
            \item Si un code déjà utilisé est scanné, l'entrée est refusée et une alerte de fraude est générée.
            \item En cas de fraude avérée, le fraudeur est expulsé et le détenteur légitime est notifié pour désactiver son ancien code QR.
        \end{itemize}

        \textbf{\textit{Limitations et Cas Particuliers}}
        \vspace{8px}

        \begin{itemize}
            \item Pendant les événements spéciaux, les portes sont ouvertes sans nécessité de code QR.
            \item La police, les pompiers ou les entités spéciales ont des accès différenciés.
            \item La principale faille est la possibilité d'utiliser le même code par plusieurs personnes.
            \item Pour améliorer la sécurité, l'authentification biométrique pourrait être envisagée.
        \end{itemize}

    \subsection{Itération Nº2}
        TODO
    \subsection{Itération Nº3}
        TODO

\section{Outils utilisés}
    TODO